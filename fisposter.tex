\documentclass[final, 14pt]{beamer}
\mode<presentation>{
  \usetheme{PHE_berlin}
}
%\usepackage[orientation=portrait,size=a0,scale=1.4,debug]{beamerposter}
\usepackage[size=custom,width=100,height=100,scale=1, debug]{beamerposter}
\usepackage[utf8]{inputenc}
%\usepackage{default}

\setbeamertemplate{caption}[numbered] % force numbered figures and tables

\usepackage[british]{babel}
\usepackage{amsmath,amsthm, amssymb, latexsym}
\usepackage{times}\usefonttheme{professionalfonts}  % times is obsolete
\usefonttheme[onlymath]{serif}
\boldmath
\usepackage{booktabs}
\usepackage{graphicx}
\graphicspath{E:/My Documents B/MSc stuff/SME/fis2013}
\usepackage{xspace}
\usepackage[overlay, absolute]{textpos}
\usepackage{tikz}
\usepackage[square, numbers]{natbib}

\linespread{1.25}


\setbeamertemplate{navigation symbols}{}
\setbeamercolor{headline}{fg=white,bg=comp_blue}

\definecolor{main_aqua}{cmyk}{0.9, 0, 0.49, 0} 

\usepackage{hyperref}
\hypersetup{urlcolor = main_aqua}

\setbeamertemplate{headline}{  
  \leavevmode

  \begin{beamercolorbox}[wd=\paperwidth]{headline}

				\centering
        \vskip2ex
        \usebeamercolor{title in headline}{\color{fg}\textbf{\LARGE{\inserttitle}}\\[1ex]}
        \usebeamercolor{author in headline}{\color{fg}\large{\insertauthor}\\[1ex]}
        \usebeamercolor{institute in headline}{\color{fg}\normalsize{\insertinstitute}\\[1ex]}
  \end{beamercolorbox}
  
  \begin{beamercolorbox}[wd=\paperwidth]{lower separation line head}
    \rule{0pt}{2pt}
  \end{beamercolorbox}
%  \end{columns}
}



%\addtobeamertemplate{headline}{}{%
%\begin{tikzpicture}[remember picture,overlay]
%\node[anchor=north east,yshift=2pt] at (current page.north east) %{\includegraphics[height=0.8cm]{white_phe_logo.png}};
%\end{tikzpicture}}

\title[AMR trends in respiratory isolates]{Trends in the antibiotic susceptibilities of \textit{Haemophilus influenzae}, \textit{Streptococcus pneumoniae} and \textit{Staphylococcus aureus} from respiratory isolates in England, Wales and Northern Ireland, 2008-2013}
\author{Simon Thelwall\inst{1}
\and
Rebecca Guy\inst{1}
\and Katherine L Henderson\inst{1}
\and Macey L Murray\inst{2}
\and Mike Sharland\inst{3}
\and Berit Muller-Pebody\inst{1} 
\and Alan Johnson\inst{1}}

\institute[Public Health England] % (optional, but mostly needed)
{
  \inst{1}%
  Department of Healthcare Associated Infections and Antimicrobial Resistance, Centre for Disease Surveillance and Control, Public Health England
  \\
  \inst{2}%
  Centre for Paediatric Pharmacy Research, Department of Practice and Policy, UCL School of Pharmacy, University College London, London, UK
  \\
  \inst{3}%
  Paediatric Infectious Diseases Unit, St George's Hospital NHS Trust, London, United Kingdom
}

\date[November. 16th, 2013]{November. 16th, 2013}

% http://cid.oxfordjournals.org/content/51/11/1258.full
% pcv 7: 4, 6B, 9V, 14, 18C, 19F, and 23F
% pcv 13: same plus 1, 3, 5, 6A, 7F, and 19A

\begin{document}
\begin{textblock}{6}(0.1,0.2)
\includegraphics[scale=1]{white_phe_logo.png}
\end{textblock}
\begin{frame}
 \begin{columns}[t]
  \column{.3\linewidth}
%  \begin{block}{Abstract}
%  \begin{description}
%    \item [Objectives] To determine the susceptibility of \textit{Haemophilus influenzae}, \textit{Streptococcus pneumoniae} and \textit{Staphylococcus aureus} to antibiotics recommended for community treatment of bacterial pneumonia.
%   \item[Methods] Antibiotic susceptibility data for \textit{H. influenzae}, \textit{S. pneumoniae} and \textit{S. aureus} from sputum isolates reported between January 2008 and June 2013 were extracted from the Public Health England microbiology database of clinically relevant infections. 
%   Data were restricted to isolates from hospital outpatients or specimens referred by General Practitioners for patients in England, Wales and Northern Ireland.
%   Multivariable Poisson regression was performed to test associations between calendar time and number of resistant isolates reported.
%   \item[Results] Data on 180 499 isolates were extracted. 
%   There was a consistent increase in the number of isolates reported over the five year period for each organism (\textit{H. influenzae} incident rate ratio 1.11 per year, p\textless0.001, \textit{S. aureus} IRR 1.04, p\textless0.001, \textit{S. pneumoniae} IRR 1.04, p\textless0.001). 
%   Resistance to the empiric treatment ampicillin/amoxicillin increased for all three organisms between 2008 and 2013 (\textit{H. influenzae} 23.4\% to 26.3\%, IRR: 1.03, p\textless0.001, \textit{S. aureus} 75.0\% to 80.2\%, IRR: 1.01, p=0.141 and \textit{S. pneumoniae} 2.1\% to 3.9\%, IRR 1.12, p=0.001). 
%   Resistance among \textit{S. pneumoniae} isolates rose from 13.4\% to 18.7\% for clarithromycin (IRR 1.10, p\textless0.001), from 11.7\% to 19.5\% for erythromycin (IRR 1.13, p\textless0.001) and from 1.9\% to 2.9\% for any recommended cephalosporin (IRR: 1.41, p=0.003).
%   \item[Conclusion] Decreasing susceptibility to recommended antibiotics for bacterial pneumonia is a cause for concern and may impact on the formulation of national guidelines. 
%  \end{description}
%  \end{block}
 \begin{minipage}[T]{.95\textwidth} % tweaks the width, makes a new \textwidth  
  \begin{block}{Introduction}
  Lower respiratory tract infection (LRTI) is a common cause of morbidity and mortality in the UK. \citep{Davies2012}
   Current guidelines for the empiric treatment of bacterial pneumonia in the community recommend either amoxicillin, or, if not tolerated, doxycycline or clarithromycin. \citep{Lim2009}
   The selection of appropriate empiric therapy requires knowledge of the most likely causative organism and its' likely susceptibility to antimicrobial agents.
   Previous work by our group found evidence for declining susceptibility to ampicillin/amoxicillin in \textit{Haemophilus influenzae}, \textit{Staphylococcus aureus} and \textit{Streptococcus pneumoniae} and declining susceptibility to macrolides among \textit{H. influenzae} isolates. \citep{Blackburn2011}
   We seek here to update the previous work to provide a current picture of trends in resistance among bacteria isolated from lower respiratory tract samples. \\
   \end{block}
  
  \begin{block}{Methods}
Data on isolates of \textit{H. influenzae}, \textit{S. aureus} and \textit{S. pneumoniae} from respiratory samples (sputum, trachea, bronchi, broncioles or alveolar lavage) reported to the LabBase2 database between January 2008 and June 2013 were extracted. 
   Data were restricted to isolates from hospital outpatient and general practice patients in England, Wales and Northern Ireland.\\
   Isolates were considered resistant if local laboratory staff had recorded the susceptibility of the isolate to a given antibiotic as intermediate or resistant. 
   Where multiple antibiotic susceptibility test results were recorded for the same isolate, a report of resistance took precedence over susceptibility. \\
   The change in the number of isolates submitted per quarter was analysed with linear regression. 
   Multivariable Poisson regression with robust standard errors was performed to test associations between calendar time and number of resistant isolates reported.
   %% Need information on interactions here. 
   Age, sex, year, calendar quarter and region were considered potential confounders and were adjusted for accordingly.
   Interactions between age and sex were tested for using likelihood ratio tests. 
   All analyses were performed using R version 3.0.0 and ArcGIS\texttrademark v10 and code can be found at \url{https://github.com/simonthelwall/fis2013}.
  \end{block}
  
\begin{block}{\textcolor{comp_blue}{R}}  
    \begin{figure}
   \includegraphics[width = \textwidth]{figure1.png}
   \caption{Trend in LRTI isolates from community samples reported to LabBase. England, Wales and Northern Ireland 2008-2013.}
   \label{fig:fig1}
   \end{figure}

\end{block}  
\end{minipage}
  
\column{0.3\linewidth}
 \begin{minipage}[T]{.95\textwidth} % tweaks the width, makes a new \textwidth
% \begin{block}{\textcolor{comp_blue}{R}}
  \begin{block}{Results}
   Between January 2008 and June 2013 180 499 isolates were reported by 142 labs. 
   Ninety six per cent (173 854) of isolates were from sputum and 125 968 (70 \%) were from patients presenting to GPs. \\
   Between January 2008 and June 2013, there was a consistent increase in the numbers of isolates reported for each organism (figure \ref{fig:fig1}).\\
Resistance to the empiric treatment ampicillin/amoxicillin increased for all three organisms between 2008 and 2013 (\textit{H. influenzae} 23.4\% to 26.3\%, IRR: 1.03, p\textless0.001, \textit{S. aureus} 75.0\% to 80.2\%, IRR: 1.01, p=0.141 and \textit{S. pneumoniae} 2.1\% to 3.9\%, IRR 1.12, p=0.001, figure \ref{fig:fig2} and table \ref{tab:table1}).\\
   Overall, 3354 (25.43\%, 95\% CI 24.69 - 26.18) of all isolates tested in 2013 were resistant to amoxicillin. \\% Need data for this. Table/figure?
   % Figure 2 resistance to amoxicillin
   % Figure 3 resistance to tetracyclines (in particular clarithromycin)
   % Figure 4 resistance to cephalosporins 
   % Trends by age and sex? What is driving increase in resistance?
   Little variation in the proportion of isolates resistant was seen by region (figure \ref{fig:maps}). 
   Resistance to clarithromycin among \textit{S. aureus} isolates was higher in the South East, South West and London regions. 
   Resitance to clarithromycin among \textit{S. pneumoniae} isolates was higher in London than in other regions. \\
   \vspace{1 cm}
     \begin{table}
   \centering
   \small%\addtolength{\tabcolsep}{-5pt}
   \begin{tabular}{rrlll}
  \hline
 &  & \multicolumn{3}{c}{Organism}\\
  \cline{3-5}
Antibiotic & Age group & \textit{H. influenzae} & \textit{S. aureus} & \textit{S. pneumoniae}\\
\midrule
\shortstack{Ampicillin/\\amoxicillin} & & 1.00 (0.90 - 1.10) & 1.00 (0.90 - 1.10) & 1.00 (0.90 - 1.10)\\
Clarithromycin & $<45$ & 1.00 (0.90 - 1.10) & 1.00 (0.90 - 1.10) & 1.00 (0.90 - 1.10)\\
 & 45 - 64 & 1.00 (0.90 - 1.10) & 1.00 (0.90 - 1.10) & 1.00 (0.90 - 1.10)\\
 & 65 - 74 & 1.00 (0.90 - 1.10) & 1.00 (0.90 - 1.10) & 1.00 (0.90 - 1.10)\\
 & $\geq 75$ & 1.00 (0.90 - 1.10) & 1.00 (0.90 - 1.10) & 1.00 (0.90 - 1.10)\\
 & Unknown & 1.00 (0.90 - 1.10) & 1.00 (0.90 - 1.10) & 1.00 (0.90 - 1.10)\\
Doxycycline & $<45$ & 1.00 (0.90 - 1.10) & 1.00 (0.90 - 1.10) & 1.00 (0.90 - 1.10)\\
 & 45 - 64 & 1.00 (0.90 - 1.10) & 1.00 (0.90 - 1.10) & 1.00 (0.90 - 1.10)\\
 & 65 - 74 & 1.00 (0.90 - 1.10) & 1.00 (0.90 - 1.10) & 1.00 (0.90 - 1.10)\\
 & $\geq 75$ & 1.00 (0.90 - 1.10) & 1.00 (0.90 - 1.10) & 1.00 (0.90 - 1.10)\\
 & Unknown & 1.00 (0.90 - 1.10) & 1.00 (0.90 - 1.10) & 1.00 (0.90 - 1.10)\\
\bottomrule  
\end{tabular}
   \caption{Incident rate ratios for annual change in susceptibility to recommended antimicrobials, England, Wales and Northern Ireland 2008-2013.
   Adjusted for calendar quarter, laboratory region, specimen source, male sex and patient age.}
   \label{tab:table1}
   \end{table} 
      \vspace{1 cm}

   \end{block}
   \vfill
  \begin{block}{Discussion}
  Over the last five years we have observed increasing resistance to the first-line antibiotics recommended for treating bacterial pneumonia in the community.
Among \textit{S. pneumoniae} isolates there have been rises in resistance to a wide range of antibiotics. 
Whether the increases in resistances observed here are due to genuine increases in resistance or whether they are due to increased testing and reporting is diffcult to determine. 
The limited regional variation in resistance is perhaps surprising given that a recent study of community LRTI and community-acquired pneumonia (CAP) among older adults found a strong North-South gradient in incidence. \citep{Millett2013}
However, analysis of regional variation resistance is limited by variation in regional reporting to LabBase. 
Resistance to antimicrobials is known to be associated with serotype in \textit{S. pneumoniae} and further examination of serotype data may provide more information on the recent rises in resistance observed in \textit{S. pneumoniae} isolates.
%  \end{block}    
   \vspace{1 cm}

\end{block}
     \vfill
\begin{block}{Conclusions}
  Increases in resistance among organisms isolated from respiratory isolates to first-line antibiotics is a cause for concern and may impact on the formulation of national guidelines.  
        \vspace{1 cm}
\end{block}
     \vfill

\end{minipage}
    
\column{.3\linewidth}
 \begin{minipage}[T]{.95\textwidth} % tweaks the width, makes a new \textwidth
  \begin{block}{\textcolor{comp_blue}{R}}
   \begin{figure}
  \includegraphics[width = \textwidth]{figure2.png}
  \caption{Trends in resistance to antimicrobials recommended for empiric community therapy of lower respiratory tract infection. England, Wales and Northern Ireland 2008-2013}
  \label{fig:fig2} 
  \end{figure} 

  \vfill   
   \begin{figure}
   \includegraphics[width = \textwidth]{maps.png}
   \caption{Proportion of tested isolates resistant to given antibiotics by region. England, Wales and Northern Ireland January 2008 to June 2013.}
   \label{fig:maps}
   \end{figure}
  \end{block}
  \vfill
    \begin{block}{References}
   \bibliographystyle{plain}
\footnotesize{\bibliography{fis2013}}
   \vfill
  \end{block}
\end{minipage}  

 \end{columns}

\end{frame}

\end{document}
