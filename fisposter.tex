\documentclass[final]{beamer}
\mode<presentation>{
  \usetheme{PHE_berlin}
}
\usepackage[orientation=portrait,size=a0,scale=1.4,debug]{beamerposter}
\usepackage[utf8]{inputenc}
\usepackage{default}

\usepackage[british]{babel}
\usepackage{amsmath,amsthm, amssymb, latexsym}
\usepackage{times}\usefonttheme{professionalfonts}  % times is obsolete
\usefonttheme[onlymath]{serif}
\boldmath
\usepackage{booktabs}
\usepackage{graphicx}
\usepackage{xspace}
\setbeamertemplate{navigation symbols}{}
\setbeamercolor{headline}{fg=white,bg=comp_blue}

\definecolor{main_aqua}{cmyk}{0.9, 0, 0.49, 0} 

\usepackage{hyperref}
\hypersetup{urlcolor = main_aqua}

\setbeamertemplate{headline}{  
  \leavevmode

  \begin{beamercolorbox}[wd=\paperwidth]{headline}
				\centering
        \vskip2ex
        \usebeamercolor{title in headline}{\color{fg}\textbf{\LARGE{\inserttitle}}\\[1ex]}
        \usebeamercolor{author in headline}{\color{fg}\large{\insertauthor}\\[1ex]}
        \usebeamercolor{institute in headline}{\color{fg}\normalsize{\insertinstitute}\\[1ex]}
  \end{beamercolorbox}
  
  \begin{beamercolorbox}[wd=\paperwidth]{lower separation line head}
    \rule{0pt}{2pt}
  \end{beamercolorbox}
}

\title[AMR trends in respiratory isolates]{Trends in the antibiotic susceptibilities of \textit{Haemophilus influenzae}, \textit{Streptococcus pneumoniae} and \textit{Staphylococcus aureus} from respiratory isolates in England, Wales and Northern Ireland, 2008-2013}
\author{Simon Thelwall\inst{1}
\and
Rebecca Guy\inst{1}
\and Katherine L Henderson\inst{1}
\and Macey L Murray\inst{2}
\and Mike Sharland\inst{3}
\and Berit Muller-Pebody\inst{1} 
\and Alan Johnson\inst{1}}

\institute[Public Health England] % (optional, but mostly needed)
{
  \inst{1}%
  Department of Healthcare Associated Infections and Antimicrobial Resistance, Centre for Disease Surveillance and Control, Public Health England
  \\
  \inst{2}%
  Centre for Paediatric Pharmacy Research, Department of Practice and Policy, UCL School of Pharmacy, University College London, London, UK
  \\
  \inst{3}%
  Paediatric Infectious Diseases Unit, St George's Hospital NHS Trust, London, United Kingdom
}

\date[November. 16th, 2013]{November. 16th, 2013}

\begin{document}
\begin{frame}
 \begin{columns}[t]
  \column{.45\linewidth}
  \begin{block}{Abstract}
  \begin{description}
    \item [Objectives] To determine the susceptibility of \textit{Haemophilus influenzae}, \textit{Streptococcus pneumoniae} and \textit{Staphylococcus aureus} to antibiotics recommended for community treatment of bacterial pneumonia.
   \item[Methods] Antibiotic susceptibility data for \textit{H. influenzae}, \textit{S. pneumoniae} and \textit{S. aureus} from sputum isolates reported between January 2008 and June 2013 were extracted from the Public Health England microbiology database of clinically relevant infections. 
   Data were restricted to isolates from hospital outpatients or specimens referred by General Practitioners for patients in England, Wales and Northern Ireland.
   Multivariable Poisson regression was performed to test associations between calendar time and number of resistant isolates reported.
   \item[Results] Data on 180 499 isolates were extracted. 
   There was a consistent increase in the number of isolates reported over the five year period for each organism (\textit{H. influenzae} incident rate ratio 1.11 per year, p\textless0.001, \textit{S. aureus} IRR 1.04, p\textless0.001, \textit{S. pneumoniae} IRR 1.04, p\textless0.001). 
   Resistance to the empiric treatment ampicillin/amoxicillin increased for all three organisms between 2008 and 2013 (\textit{H. influenzae} 23.4\% to 26.3\%, IRR: 1.03, p\textless0.001, \textit{S. aureus} 75.0\% to 80.2\%, IRR: 1.01, p=0.141 and \textit{S. pneumoniae} 2.1\% to 3.9\%, IRR 1.12, p=0.001). 
   Resistance among \textit{S. pneumoniae} isolates rose from 13.4\% to 18.7\% for clarithromycin (IRR 1.10, p\textless0.001), from 11.7\% to 19.5\% for erythromycin (IRR 1.13, p\textless0.001) and from 1.9\% to 2.9\% for any recommended cephalosporin (IRR: 1.41, p=0.003).
   \item[Conclusion] Decreasing susceptibility to recommended antibiotics for bacterial pneumonia is a cause for concern and may impact on the formulation of national guidelines. 
  \end{description}
  \end{block}
  
  \begin{block}{Introduction}
   Current guidelines for the empiric treatment of bacterial pneumonia in the community recommend either amoxicillin, or, if not tolerated, doxycycline or clarithromycin.\cite{Lim2009}   
   \end{block}
  
  \begin{block}{Methods}
   Changes in the trend in the number of isolates reported, and the proportion of isolates resistant to given antibiotics were tested using multivariable logistic regression. 
   All analyses were performed using R and code can be found at \url{https://github.com/simonthelwall/fis2013}.
  \end{block}
  
  \column{.45\linewidth}
    \begin{block}{Results}
   
  \end{block}
  
  \begin{block}{Discussion}
   
  \end{block}
  
  \begin{block}{Conclusions}
   
  \end{block}
  
  \begin{block}{References}
   \bibliographystyle{plain}
\bibliography{fis2013}
  \end{block}
 \end{columns}

\end{frame}

\end{document}
